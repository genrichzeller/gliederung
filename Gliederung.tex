\documentclass[12pt,a4paper]{article}
\usepackage[utf8]{inputenc}
\usepackage[german]{babel}
\usepackage{amsmath}
\usepackage{amsfonts}
\usepackage{amssymb}
%\usepackage{siunitx}
\usepackage[left=2cm,right=2cm,top=2cm,bottom=2cm]{geometry}


\newcommand{\ttbar}{$\text{t} \bar{ \text{t}}$}
\newcommand{\tth}{$\text{t} \bar{ \text{t}} \text H$}
\newcommand{\tthtobb}{$\text{t} \bar{ \text{t}} \text{HToBB}$}
\newcommand{\pt}{$p_{\text T}$}

\title{Gliederung:\\\textbf{Identifikation von \tth -Endzuständen mit hohen Transversalimpulsen am CMS-Experiment}}
\author{Genrich Zeller}

\begin{document}
\maketitle
\tableofcontents
\newpage
\section{Einleitung (1S.)}
\begin{itemize}
\item Problemstellung
\item Motivation
\end{itemize}
\section{Theorie (3S.)}
\subsection{Standardmodell}
\subsubsection{Higgsmechanismus}
\subsection{Higgs-Produktion in Assoziation mit einem Top-Quark-Antiquark-Paar}
\subsection{Ereignisse mit hohen transversalen Impulsen}
\section{Experiment (2S.)}
\subsection{LHC}
\subsection{CMS-Experiment}
\section{Datensätze (3S.)}
\subsection{Monte Carlo Simulation}
\subsubsection{Verwendete Monte Carlo Datensätze}
\subsection{Objektrekonstruktion}
\subsection{Ereignisselektion}
\section{Leptonisolation im Bereich hoher transversaler Energien}
\begin{itemize}
\item Kurze Einleitung (Problemstellung, Herangehensweise usw.)
\end{itemize}
\subsection{Untersuchung der Leptonisolation in Abhängigkeit verschiedener Variablen}
%\subsection{Unterschiede der $e$/$\mu$ Zerfallskanäle}
\subsection{Validierung und Optimierung des Standardschnitts}
\begin{itemize}
\item 2D Histogram \pt \_vs\_RelIso
\item Profiles/Projections
\item Validierung/Optimierung des Standardschnittes im e/$\mu$-Kanal
\end{itemize}
\subsection{Schlussfolgerung}
\section{Verhalten von Single-Lepton-Triggern im Bereich hoher transversaler Energien}
\begin{itemize}
\item Kurze Einleitung (Problemstellung, Herangehensweise usw.)
\end{itemize}
\subsection{Triggereffizienz bei Betrachtung verschiedener Kategorien (Bins?)}
\subsection{Schlussfolgerung}
\section{Zusammenfassung}
\section{Quellenangaben}
%\section{Tabellenverzeichnis}
%\section{Bildverzeichnis}
\section{Anhang}
\end{document}