\documentclass[12pt,a4paper]{article}
\usepackage[utf8]{inputenc}
\usepackage[german]{babel}
\usepackage{amsmath}
\usepackage{amsfonts}
\usepackage{amssymb}
\usepackage{siunitx}
\usepackage[left=2cm,right=2cm,top=2cm,bottom=2cm]{geometry}


\newcommand{\ttbar}{$\text{t} \bar{ \text{t}}$}
\newcommand{\tth}{$\text{t} \bar{ \text{t}} \text H$}
\newcommand{\tthtobb}{$\text{t} \bar{ \text{t}} \text{HToBB}$}
\newcommand{\pt}{$p_{\text T}$}

\title{Gliederung:\\\textbf{Identifikation von \tth -Endzuständen mit hohen Transversalimpulsen am CMS-Experiment}}
\author{Genrich Zeller}

\begin{document}
\maketitle
\section{Einleitung \hspace*{11cm} 1S}
\begin{itemize}
\item Problemstellung
\item Motivation
\end{itemize}
\section{Teilchenphysik}
\subsection{Standardmodell}
\subsection{Higgsfeld}
\subsection{\ttbar -Higgs}
\subsection{Besonderheiten im Bereich hoher Transversalimpulse}
\section{CMS-Experiment}
\subsection{Aufbau}
\subsection{?}
\section{Event-Rekonstruktion}
\subsection{Monte-Carlo Generation}
\subsection{Eventselektion}
\section{Untersuchung der relativen Isolation im semileptonischen Kanal}
\begin{itemize}
\item Kurze Einleitung (Problemstellung, Herangehensweise usw.)
\end{itemize}
\subsection{Auswertung}
\begin{itemize}
\item 2D Histogram \pt \_vs\_RelIso
\item Profiles/Projections
\item Validierung/Optimierung des Standardschnittes im e/$\mu$-Kanal
\end{itemize}
\subsection{Schlussfolgerung}
\section{Untersuchung des Leptontriggers im semileptonischen Kanal}
\begin{itemize}
\item Kurze Einleitung (Problemstellung, Herangehensweise usw.)
\end{itemize}
\subsection{Auswertung}
\subsection{Schlussfolgerung}
\section{Zusammenfassung}
\section{Quellenangaben}
%\section{Tabellenverzeichnis}
%\section{Bildverzeichnis}
\section{Anhang}
\newpage
\tableofcontents
\end{document}